\documentclass{article}

\usepackage{fancyhdr}
\usepackage[parfill]{parskip}
\usepackage{tikz}
\usepackage{amssymb,amsmath}
\usepackage{hyperref}
\usepackage{bookmark}
\makeatletter
\renewcommand\@seccntformat[1]{}
\makeatother

\pagestyle{fancyplain}

\author{Todd Davies}
\title{3.2.11 Ecology and biodiversity}
\date{\today}

\begin{document}

\rhead{3.2.11 Ecology and biodiversity}
\lhead{\today}

\maketitle

\section{Species diversity}
\thispagestyle{empty}

Species diversity is the number of species present and the abundance of those
species in a community.

A higher species diversity of one type of organism can increase the species
diversity of another type. For example, in areas with a high species diversity
of plants and trees (such as a rainforest) will also have a high species
diversity of insects, animals and birds since there will be more habitats and a
larger selection of foods.

It is often important to measure the diversity of an ecosystem over time to
identigy areas that are undergoing major change.

\subsection{Measuring species diversity}

There are a number of ways to measure species diversity, the simplest being to
count up the number of different species.

However, other measures are better since they are able to take into account the
population sizes of species too. One such method is called the index of
diversity. It gives a number that is higher for more diverse areas. The formula is as follows:

\[
	d = \frac{N(N-1)}{\sum n(n-1)}
\]

Where:

\begin{center}
\begin{tabular}{|l|l|}
	\hline
	N & The total number of organisms of all species\\ \hline
	n & The total number of one species\\ \hline
	$\sum$ & Sum of\\ \hline
\end{tabular}
\end{center}

\subsubsection{Example - species diversity}

There are three different species of flower in a field, one is red, one is white
and the other is blue.

There are eleven flowers altogether, three of them are red and and five are white.

{\bf Calculate the species diversity index of the field.}

\begin{enumerate}

	\item First we need to find the number of blue flowers.\\
		  11 - 3 - 5 = 3
	\item Now we must use the formula ($d = \frac{N(N-1)}{\sum n(n-1)}$)\\
		  \[
		  	\begin{split}
		  	d &= \frac{11(11-1)}{3(3-1) + 5(5-1) + 3(3-1)}\\
		  	  &= \frac{11\times10}{(3\times2) + (5\times4) + (3\times2)}\\
		  	  &= \frac{110}{6 + 20 + 6}\\
		  	  &= \frac{110}{32}\\
		  	  &= 3.44
		  	\end{split}
		  \]

\end{enumerate}

\subsection{Species diversity is affected by human activities}

Human activities such as agriculture and deforestation can affect species
diversity.

Deforistation reduces the number of trees (and sometimes the number of tree
species) in an environment. This has the side effect of destroying habitats and
food sources for many birds and insects, and forces them to migrate into
increasingly smaller areas of forest (which could temporarily increase species
diversity in those areas).

Agriculture works best when large areas of land are all the same, growing only
one crop. Woodland is cleared in order to make room for the crops, and hedgerows
are removed to increase the size of many small fields into fewer large fields.
Both of these acts decrease species diversity.

Pesticides and herbicides are used to kill undesirable species, which not only
reduces the population sizes of those species, but also decreases the population
size of any organisms that may rely on them.

\end{document}