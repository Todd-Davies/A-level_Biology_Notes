\documentclass{article}

\usepackage[normalem]{ulem}
\usepackage{fancyhdr}
\usepackage[parfill]{parskip}
\usepackage{tikz}
\usepackage{multicol}
\pagestyle{fancyplain}
\usetikzlibrary{shapes, patterns}

\title{Coordination}
\author{Todd Davies}
\date{\today}

\begin{document}

\rhead{Coordination}
\lhead{\today}

\maketitle

\section*{Chemical communitcation}
\thispagestyle{empty}

\subsection*{Chemical controllers}

Chemical controllers are compounds that affect the activity of other cells in an
organism. They can be classified into two groups, {\it chemical mediators} and
{\it hormones}.

\subsection*{Chemical mediators}

Chemical mediators is a signal molecule that is released by one cell to trigger
a change in neighboring cells. They cross the gap between cells by diffusion.

An example of a chemical mediator is a {\it histamine}. Histamines reside in
mast cells in humans, which when stimulated by an event such as tissue damage,
release histamines into their local tissue. Histamines have effects such as
increasing the permiability of local tissue allowing components of the immune
system (such as lymphocytes) to diffuse into the damaged area of the body and
start destroying pathogens.

The activity of histamines can be blocked by drugs called {\it antihistamines},
that act to prevent histamines from binding to histamine receptors on cells.
They can be used to reduce the effect of insect stings etc.

\subsubsection*{Prostagladins}

Prostagladins are a group of chemical mediators found in nearly all animal
tissue. They are active for only a very short time in the body before they are
broken down and so must be synthesised on demand by the cells that need them.
Synthesis takes place in the cell membrane.

Prostagladins have an effect on blood clotting, and inflamation.

\section*{Hormones}

Hormones are chemical messengers involved in nearly all aspects of the body.
They help regulate growth, metabolism and reproduction. The action of hormones
is usually widespread throughout the body and lasts a long time since they are
released by {\it endocrine glands} that secrete the hormones straight into the
blood for circulation around the body.

The main function of hormones is to alter the rate of reaction in a set of
target cells. They can do this in many ways, including:

\begin{itemize}

	\item Altering the rate of protein synthesis
	\item Changing the rate of enzyme activity
	\item Modifying cell membrane transport
	\item Introducing secretory activity

\end{itemize}

Each cell in the body has receptors for differnt hormones, which allows hormones
to affect very specific functions in the body.

Endocrine glands can be stimulated in a number of ways, either from changing
conditions in the blood, hormones in the blood (secreted by other endocrine
glands) or from nerve fibres of the autonomic nervous system.

Since endocrine glands can stimulate eachother, they form a sort of network of
loosley coupled nodes that form the hormone system.

\section*{Nerve impulses}

%TODO:Other things here

\subsection*{Factors affecting transmission speed in neurones}

%TODO: Check this
Myelinated neurones have a faster transmission speed than non-myelinated
neurones. This is because the Schwann cells that form the myelin layer prevent
the diffusion of ions. This also means that depolarisation only occurs at the
nodes of Ranvier, and the action potiential only has to jump from node to node.

Higher temperatures increase the rate of diffusion of ions, and so increases the
transmission speed.

In axons that aren't myelinated, the larger diameter of the axon, the faster the
transmission. Because of the smaller surface area to volume ratio of larger
axons, a smaller proportion of ions leak and the transmission speed is
increased.


\end{document}