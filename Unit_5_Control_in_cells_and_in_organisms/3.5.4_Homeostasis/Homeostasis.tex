\documentclass{article}

\usepackage[normalem]{ulem}
\usepackage{fancyhdr}
\usepackage[parfill]{parskip}
\pagestyle{fancyplain}

\title{Homeostatis}
\author{Todd Davies\\
	\texttt{\small todd434@gmail.com}}
\date{\today}

\begin{document}

\rhead{Homeostatis}
\lhead{\today}

\maketitle

\section*{What is homeostasis?}
\thispagestyle{empty}

The word homeostasis means {\it steady state} and referes to the body's
mechanism of keeping the conditions within itself within certain limits.

The mechanism of homeostasis can be broken down into three steps; What causes a
change in conditions (e.g. sweeties increase glucose level), what detects the
change (e.g. hypothalamus in the brain), and how is the change corrected (e.g.
insulin is released to lower glucose levels).

\subsection*{Stimulus and response}

If the body detects a stimulus (such as the blood temperature rising) it will
evoke a response (such as sweating) in order to attempt to return the property
causing the stimulus back into it's normal range.

\subsection*{Example of homeostasis: regulation of blood glucose}

\subsubsection*{Receptors}

The pancreas has two types of receptor cells ($\alpha$ and $\beta$) that monitor
the glucose levels in the blood. If the glucose level is too high, then $\beta$
cells are stimulated to release insulin. If it's too low, then $\alpha$ cells
are stimulated and secrete glucagon.

\subsubsection*{Effectors}

Insulin stimulates cells in the liver, muscles and fatty deposits to use up more
glucose by storing it in glycogen, converting it into fats or respiring faster.
Insulin works by increasing the number of carrier proteins at the surface of
cells (by moving them to the cell membrane from the cytoplasm) so that more
glucose is taken into the cells.

Glucagon stimulates the liver to break into it's reserves of glycogen and
release more glucose into the blood (glycogenolysis). If there's no glycogen
left, then glucose can be formed from pyruvate by gluconeogenesis.

{\it N.b. during starvation, the body breaks proteins to produce amino acids,
which are then used to produce pyruvate the be converted into glucose by
gluconeogenesis. This is why people become skinny and weak when starving.}

Since insulin and glucagon have opposite effects on the blood sugar level, they
are said to be {\it antagonistic}. Many other hormones in homeostasis work in
antagonistic pairs.

\subsubsection*{Another stimulus}

Sometimes we get an adrenaline rush. Both adrenaline and glucagon form weak
bonds with cell receptors and in doing so stimulate another chemical cAMP inside
the cytoplasm to initiate a faster rate of glycogenolysis. This indirect method
of stimulation is called the {\it second messanger model}.

\subsection*{Diabetes}

There are two forms of diabetes. Both result in an inability to control the
glucose level in the blood.

The first type (type I) is present from a young age, and is the result of cells
in the pancreas being destroyed by an autoimmune response.

The second type of diabetes (type II) is a result of the changes in the glucose
level not being detected by the otherwise healthy cells in the pancreas. This
type of diabetes mostly occurs in middle aged overweight people.

In both cases, the conditions that would normally invoke a response from the
body now don't. This means that the glucose levels in diabetic people must be
monitored manually and insulin injections given if needed. Technology such as
watches that monitor blood concentration is being developed, but can be
unreliable.

\end{document}
