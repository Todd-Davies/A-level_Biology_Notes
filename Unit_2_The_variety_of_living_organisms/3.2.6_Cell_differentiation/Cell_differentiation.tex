\documentclass{article}

\usepackage{fancyhdr}
\usepackage[parfill]{parskip}
\usepackage{tikz}

\pagestyle{fancyplain}

\author{Todd Davies}
\title{3.2.6 Cell differentiation}
\date{\today}

\begin{document}

\rhead{3.2.6 Cell differentiation}
\lhead{\today}

\maketitle

\section*{Cell differentiation}
\thispagestyle{empty}

In order to perform specific functions, organisms must have a variety of
different cell types. Each cell type is specialised and can perform one or more
specific function(s). The process of becoming specialised is called
differentiation.

\subsection*{Cell organisation}

Similar cells are organised into tissues. For example, one muscle cell won't be
able to generate much of a force, but one million of them in a muscle tissue can
work togehter to create a large force. Tissues aren't always made of only one
cell, but all the cells usually work together to perform a function.

Tissues are organised into organs. A leaf for example is made of the following
tissues:

\begin{itemize}

	\item {\bf Lower epidermis} - contains pores to allow the exchange of gas.

	\item {\bf Spongy mesophyll} - full of space to let gases circulate.

	\item {\bf Palisade mesophyll} - photosynthesis occurs in these cells.

	\item {\bf Xylem} - carries water to the leaf.

	\item {\bf Phloem} - carries sugar away from the leaf.

	\item {\bf Upper epidermis} - covered in a waterproof waxy cuticle to reduce water loss.

\end{itemize}

Organs are then organised into systemd. Each system has a particular function,
examples include:

\begin{itemize}

    \item {\bf Circulatory system} - facilitates transport of substances within
	the body.

    \item {\bf Respiratory system} - brings oxygen into the body and removes
	carbon dioxide.

	\item {\bf Digestive system} - digests food into substances the body can
	use.

\end{itemize}


\end{document}