\documentclass{article}

\usepackage{fancyhdr}
\usepackage[parfill]{parskip}

\pagestyle{fancyplain}

\author{Todd Davies}
\title{3.2.2 DNA}
\date{\today}

\begin{document}

\rhead{3.2.2 DNA}
\lhead{\today}

\maketitle

\section*{What is DNA}
\thispagestyle{empty}

DNA is an information carrying molecule. The information in the molecule
determines the structure of proteins. DNA is a polynucleotide, made of many {\it
nucleotides} joined together.

\subsection*{What is a nucleotide?}

A nucleotide is a molecule that contains a pentose \marginpar{Pentose is a 5
carbon sugar} sugar, phosphate group and nitrogenous base.

Each nucleotide contains the same sugar (deoxyribose) and phosphate group, but
there can be four possible bases;

\begin{itemize}

	\item Adenine (A)

	\item Thymine (T)

	\item Cytosine (C)

	\item Guanine (G)

\end{itemize}

\subsection*{How is DNA structured?}

DNA is made of two strands of complementary nucleotides that are stuck together
with hydrogen bonds.

Individual nucleotides join together to form a sugar-phosphate backbone by
bonding the phosphate group of one nucleotide to the deoxyribose sugar of
another.

Two strands of DNA are joined together by hydrogen bonds forming between their
bases. Each base can only join with it's partner \marginpar{A is complementary
to T, while C is complementary to G}. This is called {\it complementary base
pairing}

DNA naturally winds up into a double helix shape.

\subsection*{Why is DNA structured as it is?}

Organisms need to have lots (gargantuan amounts) of DNA in order to store all
the information they need to live, grow, reproduce etc. The tightly coiled
structure of DNA allows it to take up as small space as possible.

The fact that each molecule of DNA has a paired structure means it's easy to
copy, which makes cell division easier.

The double-helix structure of DNA makes it strong and stable.

\section*{Where is DNA stored?}

Even though the structure and interpretation of DNA is identical in any orgnism,
different types of organisms store DNA differently.

\paragraph*{How do eukaryotic organisms store DNA?}

\marginpar{Eukaryotic cells are found in plants and animals}

Eukaryotic organisms store DNA in linear strands that exist in chromosomes. Each
chromosome contains one long thread of DNA. The chromosome is wound around a
protein (called a histone) which helps to support the DNA molecule.

\paragraph*{How do prokaryotic organisms store DNA?}

The DNA in prokaryotes is shorter and is arranged into circular {\it plasmids}.
These are loops of DNA that coil up many times to fit into the cell.

\section*{Genes}

A gene is a section of DNA that codes for a protein. Each protein is a
polypeptide made from a long chain of amino acids. The order of nucleotide bases
on the gene determines the order of amino acids in the protein, which in turn
determines the structure of the protein.

Each amino acid in the gene is represented by three bases called {\it a
triplet}. Each amino acid can be represented by more than one triplet. Because
of this, the code of DNA is sometimes refered to as a degenerate code.

Because the sequence of bases on genes dictates the structure of proteins, a
mutation in a gene can result in large changes in the structure of the protein
coded for by that gene, which will usually result in a non-funcional protein.

This is because the function of proteins are very much to do with their
structure. For example, the active site of an enzyme is specific to the shape of
a substrate. If the active site changes (because the structure of the protein
has changed) then it will no longer be able to form an {\it enzyme-substrate
complex} and the enzyme won't work as desired.

\subsection*{Introns and Exons}

Sections of DNA can be broadly split into two classes; Introns and Exons.

Exons are sections of DNA that actually code for proteins (genes in other
words).

However, introns are sections of DNA that don't code for proteins. They are
removed from the DNA strand before protein synthesis, and their purpose isn't
known. They are often characterised by long sections of one base, or sections of
many repeats over and over.

\subsection*{Alleles}

There is often more than one type of gene. Different forms of the same gene are
called Alleles.

The base sequence of each allele is slightly different, though they always exist
on the same part of the same chromosome (known as a {\bf locus}).

Different alleles produce different versions of the same characteristic, for
example eye colour or blood type.

\section*{Meiosis}

Meiosis is a type of cell division. It is the process where a diploid cell
\marginpar{Diploid of course, means having twice the normal number of
chromosomes ($2n$)} splits into two to form two haploid cells.
\marginpar{Haploid of course, means the cell has the normal amount of
chromosomes ($n$)}.

Meiosis occurs in cells that have just undergone fertilisation. When two haploid
gamates fuse (an egg and a sperm for example). This ensures that the resulting
organism can have {\bf genes from both parents} but still have the correct
number of chromosomes.

Here's exactly how meiosis works:

\begin{itemize}

	\item The DNA unravels and replicates so that there are $4n$ chromosomes.

	\item The DNA morphs to form double armed chromatids from two sister
	chromatids, joined in the middle by a centromere.

	\item The cell divides once (Meiosis I) and the chromosomes arrange
	themselves into homologous (like) pairs. This brings the number of
	chromosomes in each cell to $2n$ again.

	\item The cell then divides again (Meiosis II) and the pairs of sister
	chromatids are seperated so that one 'arm' goes into each cell.

	\item The four cells that are produced are all genetically different from
	each other, since they all have different combinations of alleles.

\end{itemize}

\section*{Genetic variation}

Genetic variation is really important. Without it, we would all be genetically
identical, which would make us far more suseptible to mass extinction.

\subsection*{Crossing over}

During Meiosis I, when the sister chromosomes pair up, they actually twist
around each other. In doing so, little bits of them swap over. By doing this,
the genes on the chromosome are preserved, but the combinations of alleles are
swapped and variation occurs.

\subsection*{Independent segregation}

During meiosis, there are different combinations of how the chromosomes can
combine when they pair up, and so each time meiosis occurs, different
combinations of the maternal and paternal chromosomes go into the new cells,
which results in variation.



\end{document}