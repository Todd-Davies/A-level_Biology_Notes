\documentclass{article}

\usepackage[normalem]{ulem}
\usepackage{fancyhdr}
\usepackage[parfill]{parskip}
\usepackage{array}
\usepackage{longtable}
\usepackage{booktabs}
\usepackage{calc}
\pagestyle{fancyplain}

\title{Recombinant DNA technology}
\author{Todd Davies}
\date{\today}

\begin{document}

\rhead{Recombinant DNA technology}
\lhead{\today}

\maketitle

\section*{What is recombinant DNA tecnhology?}
\thispagestyle{empty}
Recombinant DNA is the general name for taking a piece of one DNA, and combining
it with another strand of DNA. This new strand of DNA (now called rDNA) can be introduced into a cell to alter the cell's genotype.

\subsection*{Sample use cases of rDNA technology}
\subsubsection*{Insulin synthesis}
Insulin can be made by E.coli. Scientists isolated the gene for insulin
production and inserted it into a plasmid normally present in an E.coli
bacterium. The plasmid was then taken up by the E.coli cell. This meant that the bacterium now started to secrete human insulin, which could be harvested. The bacteria is then cultured in large fermenters that allow scientists to extract
large quantities of human insulin.

\subsubsection*{AAT synthesis}
AAT (alpha 1 antitrypsin protein) is a protein produced by humans that helps
protect body tissue from the harmful effects of other proteins. Some people have
a genetic defect that means that not enough AAT is produced, which leads to emphysema due to enzymes destroying alveoli walls and reducing their elastcity. Scientists added DNA that produces AAT to the embryo of a female sheep, which
they then cloned many times. The protein is then extracted from the milk of
these transformed sheep to use as a treatment for AAT deficiency induced
emphysema. {\it N.b. this won't treat smoking induced emphysema since the same disease is brought about via a different cause.}

\subsubsection*{Cystic fibrosis treatment}
Cystic fibrosis is caused by a mutation in the human CTFR gene. The CTFR gene
codes for a protein that transports chloride across epithelial cell membranes.
When the CTFR gene is mutated, the tertiary structure of the CTFR protein
changes, so no chloride ions can be actively transported out of epithelial
cells. If chloride ions {\it were} transported (like they are in normal cells),
then the water potential of the lumen would decrease so that water would pass
out of the cell by osmosis. In people with cystic fibrosis this doesn't happen
since the CTFR protein doesn't work and the fluid in the lumen becomes very
viscous. 

It is possible to extract healthy CTFR genes from 'normal' people, but in order
to treat a patient with cystic fibrosis, these genes must be inserted into each
of the billions of cells that have CTFR proteins in the patients body. It is
possible to do that in a number of ways:
\begin{longtable}{p{\textwidth-20\tabcolsep-1in} p{\textwidth-20\tabcolsep-1in}}
Method & Issues\\ \midrule
\endfirsthead
Method & Issues\\ \midrule
\endhead
\midrule
\multicolumn{2}{r}{continued \ldots}
\endfoot
\endlastfoot
	Wrap healthy CTFR genes in lipids that are absorbed into the patient's
	cells through the cell wall. & Most genes that are absorbed aren't expressed
	by the cell, so few cells are made to function\\ \midrule
	Insert healthy CTFR genes into harmless viruses that then infect the
	patient's epithelial cells and release their payload of CTFR gene into the
	cell. & The viruses released don't harm the cell or replicate (in theory,
	though there have been cases in trials where the viruses did act as
	pathogens), but there is only a slim chance that the deposited genes will
	get into the nucleus and start being used to create proteins.\\ \midrule
\end{longtable}
\end{document}