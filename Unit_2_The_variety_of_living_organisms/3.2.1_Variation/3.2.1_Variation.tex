\documentclass{article}

\usepackage{fancyhdr}
\usepackage[parfill]{parskip}

\pagestyle{fancyplain}

\author{Todd Davies}
\title{3.2.1 Variation}
\date{\today}

\begin{document}

\rhead{3.2.1 Variation}
\lhead{\today}

\maketitle

\section*{What is variation?}
\thispagestyle{empty}

Variation is the differences that exist between individual organisms.

There are two types of variation; interspecific and intraspecific.

\subsection*{Interspecific variation}

This variation exists between different species. For example, turtles vary from
tortioses, and mice vary from worms.


\subsection*{Intraspecific variation}

This is the variation occuring within a species. Examples include eye colour,
height, fur colour and skin type.

Intraspecific variation is caused by both genetic and environmental factors as
described below:

\paragraph*{Genetic factors}

\marginpar{{\bf Phenotype}: Observable features or characteristics of an organism}

\marginpar{{\bf Genotype}: The alleles possessed by an organism}

Even though all members of one species share the same genes, each individual
will have a different set of alleles, which make up the genotype of that
individual. Variation in the genotype of individuals in a population also
results in variations in the phenotype. Since the genes of an organism are
passed down from it's parents, genetic variation is inherited.

\paragraph*{Environmental factors}

The phenotype of an individual can be related to it's environment. Abiotic
conditions can change the growth rates of organism, or it's resistance to
disease. Identical twins are genetically identical, but always have minor
differences such as slightly different accents, or perhaps a scar.

\paragraph*{Causes of a specific variation}

When identifying exactly why an organism has a specific trait, there are often
lots of misleading factors. For example, the level of antioxidants in berries is
related to both the species of the berry and the amount of minerals in the soil.

\section*{Investigating variation}

In order to study variation, a population must be sampled. A sample is a small
portion of a population that is representative of the population as a whole.

To make sure that the sample accurately represents the population, we must make
sure that the individuals selected for sampeling are picked at random as to
avoid bias.

In order to ensure the conclusions we draw from the sample, we must be able to
statistically verify that the results aren't just due to chance.

\subsection*{Statistical analysis}

\marginpar{The {\bf standard deviation} is a measure of the spread of values
about the mean.}

We can compare the {\it mean} of two samples to see if they differ. We can also
use {\it standard deviation} (SD) to find how much a sample varies, and to see
if the two samples differ significantly.

\end{document}