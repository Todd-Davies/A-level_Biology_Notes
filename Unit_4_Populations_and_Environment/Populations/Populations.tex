\documentclass{article}

\usepackage[normalem]{ulem}
\usepackage{fancyhdr}
\usepackage[parfill]{parskip}
\pagestyle{fancyplain}

\title{Populations}
\author{Todd Davies}
\date{\today}

\begin{document}

\rhead{Populations}
\lhead{\today}

\maketitle

\section*{Definitions to learn}
\thispagestyle{empty}
\begin{itemize}
	\item \textbf{Habitat} - The place where an organism lives.
	\item \textbf{Population} - All the organisms of one species in a habitat.
	\item \textbf{Community} - Populations of different species in a habitat make up a community.
	\item \textbf{Ecosystem} - All the organisms living in a particular area and all the non living (abiotic) conditions.
	\item \textbf{Abiotic conditions} - The non-living features of the ecosystem.
	\item \textbf{Biotic conditions} - The living features of the ecosystem.
	\item \textbf{Niche} - The role of a species within it's habitat.
	\item \textbf{Adaptation} - A feature that members of a species have that increases their chance of survival and reproduction.
\end{itemize}

\section*{Niches}
The niche that a species occupies includes all the biotic and abiotic interactions it makes with it's environment. Each species has it's own niche, one niche can only be occupied by one species.

If two species try and occupy the same niche, they will compete with each other until only one species is left.

\section*{Adaptations}
Adaptations are features that increase an organisms chance of survival and reproduction. They can be physical (such as two stomachs or big ears) or behavioural (such as mating rituals). Organisms that are better adapted to their environment are more suited to their environment which means they are more likley to survive and pass on their genes to the next generation.

\section*{Investigating populations}
There are two ways of quantifying a population, by abundance and by distribution.

\textbf{Abundance} is the number of individuals of one species in a particular area. It can be measured by using percentage cover, or the number samples that contained organisms (e.g. 70\% of samples).

\textbf{Distribution} is the location of the species you are investigating.

\subsection*{Methods of investigating species}
\subsubsection*{Insects \& animals}
\begin{itemize}
	\item \textbf{Pitfall traps} are containers that organisms fall into and cannot get out of. They must protect the contents from any predators.
	\item \textbf{Humane traps} are used to trap mammals and larger reptiles. They have a door that closes when the animal has been tempted inside by bait.
	\item \textbf{Mark release recapture} is a technique used to measure the abundance of a mobile species. You catch a group of animals from a population, mark them and release them back into the wild. Then you catch some more and see how many of the marked ones were caught again. To work out an estimate of population size, you use:
	\[
		\textrm{Total population size} = \frac{\textrm{number caught in sample 1} \times \textrm{number caught in sample 2}}{\textrm{number marked in sample 2}}
	\]
\end{itemize}
\subsubsection*{Plants}
\begin{itemize}
	\item \textbf{Quadrats} are used to measure the species frequency of a number of individual spots. This makes it easy to find percentage cover.
	\item \textbf{Transects} You can use a transect (a line) to measure how the diversity of species change over an area. You can have line transects that measure the species touching the line, belt transects that use quadrats along the line, or interrupted transects that take measurements at intervals.
\end{itemize}

\subsubsection*{Ethical issues}
Some field work may harm the environment and the species living there. For example, the mark release recapture technique could put marked organisms at a disadvantage if they are marked improperly, or a quadrat may harm a delicate plant.

\section*{Population sizes}
The population size is the total number of organisms of one species in a habitat. The population size can vary based on any abiotic and biotic factors. When these factors are ideal, the population grows fast as the organisms reproduce successfully.

Interspecific competition is when organisms of different species compete with each other. This means the resources available to both species are shared and so they both get less. If there are two competing species, then one is likely to be outcompeted.

Intraspecific competition is when the population of a species is too big and there aren't enough resources for all of the organisms, so the population starts to decline.

Predator and prey populations are linked, since when there's lots of prey, the predators can reproduce lots as there is lots of food, but this decreases prey numbers since they are predated, which then decreases predator numbers as they begin to starve. This makes the populations fluctuate.

\section*{Human populations}
Population growth is calculated using birth and death rates (these are usually per 1000 people per year). The growth rate is calculated as:
\[
		\textrm{birth rate - death rate}
\]
To get a growth percentage, you do:
\[
	\textrm{Growth rate (percent)} = \frac{\textrm{growth rate}}{1000} \times 100
\]

\subsection*{The demographic transition model}
The DTM shows changes in birth rate, death rate and population size over time. It is divided into the following five stages:
\begin{enumerate}
	\item[] \textbf{Stage 1} BR and DR are high and fluctuating. Population size is low. Health care, contraception and sanitation are usually non existent. E.g. Amazon tribes
	\item[] \textbf{Stage 2} DR falls as health care and sanitation improve. BR is high as there is no family planning. Population starts to increase. E.g. Afghanistan
	\item[] \textbf{Stage 3} BR falls as family planning becomes popular. DR falls more since health care and sanitation improves. Population increases, but at a slower rate. E.g. Rural China
	\item[] \textbf{Stage 4} BR and DR are low and fluctuate. Population is stable and high. E.g. Great Britain
	\item[] \textbf{Stage 5} BR is falling and DR is rising/constant. Population ages and so people are supporting many elderly relatives. E.g. Germany
\end{enumerate}

\subsection*{Plotting human populations}
\subsubsection*{Population growth curves}
Plots the population size against time (DTM is one). They can show whether the population is increasing or decreasing by the direction of the curve. The gradient of the curve is indicative of how fast the population is changing.

\subsubsection*{Survival curves}
Shows the percentage of individuals born that are alive at any given age by plotting percentage of survivors against age. We can use these to calculate the average age people die and so work out life expectancy.

\subsubsection*{Population pyramids}
Plots the number of people alive on the x axis and their age on the y axis. Negative x is usually for females and positive x is usually for males.
\end{document}
