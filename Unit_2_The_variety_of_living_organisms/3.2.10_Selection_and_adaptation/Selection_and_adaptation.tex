\documentclass{article}

\usepackage{fancyhdr}
\usepackage[parfill]{parskip}
\usepackage{tikz}
\usepackage{hyperref}
\usepackage{bookmark}
\makeatletter
\renewcommand\@seccntformat[1]{}
\makeatother

\pagestyle{fancyplain}

\author{Todd Davies}
\title{3.2.10 Selection and adaptation}
\date{\today}

\begin{document}

\rhead{3.2.10 Selection and adaptation}
\lhead{\today}

\maketitle

\section{Antibiotics}
\thispagestyle{empty}

Antibiotics are chemicals that are able to kill or inhibit the growth of
bacteria. Different types of antibiotics are able to target bacteria in
different ways.

Some antibiotics prevent the growth of bacteria by inhibiting the formation of
the cell wall (an organelle not found in animals). This can lead to {\it osmotic
lysis} because the weakened cell wall may not be able to stand the difference in
pressures between the inside and the outside of the cell and may burst as a
result.

\section{Antibiotic resistance}

Some bacteria have developed resistance to antibiotics over time. This is
because random mutations in the bacterial DNA have caused proteins to be
produced that aren't affected by the antibiotics that target them, and so the
bacteria is resisistant to the antibiotic. As a concequence, the resistant
bacteria aren't killed by the antibiotic and go on to reproduce, which confers a
selective advantage to bacteria with the mutation.

Restistance to antibiotics can be passes on both vertically and horizontally.

\subsection{Vertical gene transmission}

Vertical gene transmission occurs when a bacteria with a mutated gene reproduces
and produces two new daughter cells, with identical DNA. The gene is now in two
bacteria cells.

\subsection{Horizontal gene transmission}

Bacteria can join together using a {\it conjugation tube}. This allows them to
send copies of their plasmids between each other which facilitates gene
transmission. This can occur between bacteria of the same species or even
between bacteria of different species.

\end{document}