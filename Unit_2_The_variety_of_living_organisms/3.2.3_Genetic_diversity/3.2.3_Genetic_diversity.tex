\documentclass{article}

\usepackage{fancyhdr}
\usepackage[parfill]{parskip}

\pagestyle{fancyplain}

\author{Todd Davies}
\title{3.2.3 Genetic diversity}
\date{\today}

\begin{document}

\rhead{3.2.3 Genetic diversity}
\lhead{\today}

\maketitle

\section*{Genetic diversity}
\thispagestyle{empty}

Genetic diversity is the measure of how much a population's gene pool varies.

The actual base sequence of DNA in a population varies very little (around
0.5$\%$), but there can be a great variety in the alleles of genes. It only
takes a small mutation to create a different allele that can have a large change
in the phenotype of an organism. \marginpar{Remember, DNA varies greatly between
different species though.}

Genetic diversity in a population increases slowly over time as mutations occur
in DNA and produce new alleles. Genetic diversity in a population can also
increase when individuals from a different population migrate into it.

\subsection*{Genetic bottlenecks}

A genetic bottleneck occurs when a large portion of a population is killed off
leaving only a few survivors. This reduces the number of alleles in the
population and diversity is decreased. This means that the offspring of the
survivors will all have similar genes too.

\subsection*{Founder effect}

If just a few organisms migrate and start a colony, the reulting population will
have a very small gene pool and low genetic diversity. This can result in a
higher incidence of genetic disease.

Causes of the founder effect could be geographical seperation, religious stigma
or other factors.

\subsection*{Selective breeding}

Humans often breed organisms together to encourage a specific trait. This can
have the effect of reducing genetic diversity since alleles that are undesirable
are selected out. This results in a type of genetic bottleneck since the number
of alleles in the gene pool is reduced.

Selective breeding is good because it can create high yielding organisms that
are drug resistant and able to cope with harsh conditions such as cold. However,
it's also bad since it can cause a short life expectancy (if the yield of the
organism is so high that lots of stress is on the organism's body) or create a
high rate of genetic disease and suseptibility to new diseases within the
population.

\end{document}