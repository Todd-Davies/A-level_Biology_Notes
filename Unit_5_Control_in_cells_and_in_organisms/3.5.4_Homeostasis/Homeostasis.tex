\documentclass{article}

\usepackage[normalem]{ulem}
\usepackage{fancyhdr}
\usepackage[parfill]{parskip}
\pagestyle{fancyplain}

\title{Homeostatis}
\author{Todd Davies}
\date{\today}

\begin{document}

\rhead{Homeostatis}
\lhead{\today}

\maketitle

\section*{What is homeostasis?}
\thispagestyle{empty}

The word homeostasis means {\it steady state} and referes to the body's
mechanism of keeping the conditions within itself within certain limits.

The mechanism of homeostasis can be broken down into three steps; What causes a
change in conditions (e.g. sweeties increase glucose level), what detects the
change (e.g. hypothalamus in the brain), and how is the change corrected (e.g.
insulin is released to lower glucose levels).

\subsection*{Stimulus and response}

If the body detects a stimulus (such as the blood temperature rising) it will
evoke a response (such as sweating) in order to attempt to return the property
causing the stimulus back into it's normal range.

\subsection*{Example of homeostasis: regulation of blood glucose}

\subsubsection*{Receptors}

The pancreas has two types of receptor cells ($\alpha$ and $\beta$) that monitor
the glucose levels in the blood. If the glucose level is too high, then $\beta$
cells are stimulated to release insulin. If it's too low, then $\alpha$ cells
are stimulated and secrete glucagon.

\subsubsection*{Effectors}

Insulin stimulates cells in the liver, muscles and fatty deposits to use up more
glucose by storing it in glycogen, converting it into fats or respiring faster.
Insulin works by increasing the number of carrier proteins at the surface of
cells (by moving them to the cell membrane from the cytoplasm) so that more
glucose is taken into the cells.

Glucagon stimulates the liver to break into it's reserves of glycogen and
release more glucose into the blood (glycogenolysis). If there's no glycogen
left, then glucose can be formed from pyruvate by gluconeogenesis.

{\it N.b. during starvation, the body breaks proteins to produce amino acids,
which are then used to produce pyruvate the be converted into glucose by
gluconeogenesis. This is why people become skinny and weak when starving.}

Since insulin and glucagon have opposite effects on the blood sugar level, they
are said to be {\it antagonistic}. Many other hormones in homeostasis work in
antagonistic pairs.

\subsubsection*{Another stimulus}

Sometimes we get an adrenaline rush. Both adrenaline and glucagon form weak
bonds with cell receptors and in doing so stimulate another chemical cAMP inside
the cytoplasm to initiate a faster rate of glycogenolysis. This indirect method
of stimulation is called the {\it second messanger model}.

\subsection*{Diabetes}

There are two forms of diabetes. Both result in an inability to control the
glucose level in the blood.

The first type (type I) is present from a young age, and is the result of cells
in the pancreas being destroyed by an autoimmune response.

The second type of diabetes (type II) is a result of the changes in the glucose
level not being detected by the otherwise healthy cells in the pancreas. This
type of diabetes mostly occurs in middle aged overweight people.

In both cases, the conditions that would normally invoke a response from the
body now don't. This means that the glucose levels in diabetic people must be
monitored manually and insulin injections given if needed. Technology such as
watches that monitor blood concentration is being developed, but can be
unreliable.

The importance of maintaining a constant glucose level cannot be understated.
Too much glucose can lead to dehydration from an increased water potiential of
the blood, and too little glucose can lead to brain cells not being able to
respire since they can only respire glucose and not other compounds such as
lipids.

\section*{Temperature control}

Reptiles and mammals have very different temperature control methods, and very
different behaviours at different temperatures.

Reptiles are an example of an {\bf ectotherm}. Their body temperature is
dependant on the temperature of their environment. Because of this, ectotherms
are adapted to behave according to the temperature of their environment. They
often show the following behaviours:

\begin{itemize}

	\item Exposing themselves to the sun when it's nice and warm.

	\item Hiding from the sun when it's a bit too warm.

	\item Pressing their bodies onto warm ground or lifting themselves of too
	hot ground to warm up, or to prevent overheating.

	\item Generating heat from metabolising. Note that though this isn't an
	ectotherm's primary source of heat, they still metabolise (though not
	particularly fast).

	\item Ectotherms are often adapted so that their skin colour will either be
	dark to absorb as much radiation as possible, or be white to reflect as
	much radiation as possible.

\end{itemize}

{\bf Endotherms} are organisms such as mammals that internally regulate their
body temperature. Their bodies are highly adapted to function within specific
limits of temperature. Just like ectotherms, endotherms also change their
behaviour according to their temperature, maybe curling up if cold, or hiding
from the sun if hot, but they also employ a range of other mechanisms for
keeping their body temperature constant. These include:

\begin{itemize}

	\item Arteries can be dilated and constricted ({\bf vasodilation} and {\bf
	vasoconstriction}) to divert blood to or away from the skin (or any other
	organ for that matter). If the blood is diverted away from the skin, it
	passes under the layer of fat that lines the skin which helps insulate it
	from the cold.

	\item{\bf Shivering} engages the muscles in involuntary contractions which
	produces methabolic heat.

	\item Tiny muscles called {\bf hair erectors} on the skin can contract and
	make the hairs stick up to trap a layer of static air which acts as
	insulation. The opposite happens when endotherms are too hot.

	\item The {\bf metabolic rate} increases when organisms are cold so that
	more heat is produced.

	\item When an endotherm is too hot, it begins to {\bf sweat} which then
	evaporates and cools the skin. If the organism is too cold, sweating stops.
	Some organisms also pant in order to loose heat.

	\item All endotherms employ {\bf behavioural mechanisms} in order to
	conserve or loose heat such as by basking in the sun and huddling or hiding
	in the shade.

\end{itemize}

\section*{Hypothalamus - the homeostasis control center}

The hypothalamus is a portion of the brain associated with homeostasis. The
blood going the hypothalamus is monitored for the level of glucose, it's
temperature and other indicators so the hypothalamus can release hormones to
affect a change in these variables. This is an example of a {\bf negative
feedback loop} since homeostasis is achieved by a system of continual monitoring
and the subsequent release of hormones to restore abnormal variables to within
normal limits.

%TODO: Questions on homeostasis

\end{document}
