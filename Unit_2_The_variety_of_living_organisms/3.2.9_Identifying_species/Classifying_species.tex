\documentclass{article}

\usepackage{fancyhdr}
\usepackage[parfill]{parskip}
\usepackage{tikz}

\pagestyle{fancyplain}

\author{Todd Davies}
\title{3.2.9 Classifying species}
\date{\today}

\begin{document}

\rhead{3.2.9 Classifying species}
\lhead{\today}

\maketitle

\section*{Methods of classification}
\thispagestyle{empty}

There are three main methods of classification; DNA comparison, protein
comparison and comparison of behaviour.

\subsection*{DNA comparison}

Similarities and differences in the genes of different organisms can help to
place them into different taxonomic groups. DNA similarity can be measured using
DNA hybridisation or by looking at the exact order of bases.

\subsubsection*{DNA hybridisation}

The process of DNA hybridisation is as follows:

\begin{enumerate}
	
	\item DNA from two different species is collected, seperated into different
	strands and mixed together.

	\item Where the base sequences of the DNA are the same, hydrogen bonds form
	by specific base pairing. More hydrogen bonds forming indicates a stronger
	similarity between the DNA sequences.

	\item The DNA is then heated again until the DNA seperates again. The higher
	the temperature at which the DNA seperated, the more hydrogen bonds there
	were and so the more alike the two DNA samples were.

\end{enumerate}

\subsection*{Protein comparison}

Since proteins are coded for by genes, they can give an indication of
relationships between organisms too.

Proteins can be compared by examining the exact sequence of amino acids that
they are composed of, or by immunological comparisons.

\subsubsection*{Immunological comparisons}

Similar proteins bind to the same antibodies. If two species are very closely
related, antibodies produced by one species will bind to proteins of another.

\subsection*{Behavioural comparison}

Courtship behaviour displayed by some species is designed to attract an
appropriate mate. An approriate mate must be able to provide fertile offspring,
and so must be of the same species as the individual displaying the courtship
behaviour.

Examples of courtship behaviour include:

\begin{itemize}

	\item Fireflies giving off sequenced pulses of light.

	\item Crickets making sounds that are specific to their species.

	\item Male peacocks displaying their colourful tails, a pattern only found
	in peacocks.

	\item Male butterflies produce chemicals specific to their species to
	attract females of their species.

\end{itemize}

\end{document}