\documentclass{article}

\usepackage[normalem]{ulem}
\usepackage{fancyhdr}
\usepackage[parfill]{parskip}
\pagestyle{fancyplain}

\title{Succession and conservation}
\author{Todd Davies}
\date{\today}

\begin{document}

\rhead{Succession and conservation}
\lhead{\today}

\maketitle

\section*{Succession}

\subsection*{What is ecological succession?}
\thispagestyle{empty}
Ecological succession is the term used to describe the process by which an ecological community undergoes a predictable and orderly set of changes throughout the colonisation of a habitat.

\subsection*{The generic process of succession}
The process of succession can be broken down into a number of simple steps:
\begin{enumerate}
	\item A new habitat becomes available, this may be due to the formation of new land (say from a volcanic eruption that has spewed out magma), a forest fire, the creation of a road or even the formation of a sand dune on the beach.
	\item The habitat is initially very harsh, it has few resources and has no living species on it.
	\item \textbf{Pioneer species} invade the new habitat and begin to colonise it. In doing so, they also increase the richness of the habitat by leaving the nutrients they have collected when they die.
	\item Since there are already pioneer species on the habitat, other species move in, taking advantage of the ecological niche presented to them by the colonisation of the pioneer species.
	\item These new species again alter the abiotic conditions the habitat, allowing less specialised species to move in.
	\item The process repeats until a \textbf{climax community} has developed and the species in the habitat are able to keep their local environment stable.
\end{enumerate}

\subsection*{Why does succession occur?}
Succession is a natural phenomenon that occurs pretty much everywhere. It happens when organisms change their environment enough so that different species can move into a new ecological niche created by the changes brought about by the current species. The invading species then out-compete the current species and the process starts again.

I particularly like this explanation of why succession occurs:

\begin{quotation}	
	Every species has a set of environmental conditions under which it will grow and reproduce most optimally. In a given ecosystem, and under that ecosystem's set of environmental conditions, those species that can grow the most efficiently and produce the most viable offspring will become the most abundant organisms. As long as the ecosystem's set of environmental conditions remains constant, those species optimally adapted to those conditions will flourish. The "engine" of succession, the cause of ecosystem change, is the impact of established species have upon their own environments. A consequence of living is the sometimes subtle and sometimes overt alteration of one's own environment. The original environment may have been optimal for the first species of plant or animal, but the newly altered environment is often optimal for some other species of plant or animal. Under the changed conditions of the environment, the previously dominant species may fail and another species may become ascendant.\footnote{http://www.psu.edu/dept/nkbiology/naturetrail/succession.htm}
\end{quotation}

\section*{Conservation}
\subsection*{What is conservation?}
Conservation attempts to maintain biodiversity. This involves maintaining a variety of different habitats, which can between them support a wider range of species than a single type of habitat could do.

Conservation frequently involves preventing succession. For example, if a field was left alone for a long time, it would gradually become a wood. This would make some species extinct in the local area. To avoid this, farmers have sheep graze on the field so that succession doesn't progress too far.

\end{document}