\documentclass{article}

\usepackage[normalem]{ulem}
\usepackage{fancyhdr}
\usepackage[parfill]{parskip}
\usepackage{array}
\usepackage{longtable}
\usepackage{booktabs}
\usepackage{calc}
\pagestyle{fancyplain}

\title{3.5.8 - Gene cloning \& transfer}
\author{Todd Davies}
\date{\today}

\begin{document}

\rhead{3.5.8 - Gene cloning \& transfer}
\lhead{\today}

\maketitle

\section*{What is gene cloning \& transfer?}
\thispagestyle{empty}

Gene cloning refers to a method of isolating a gene in the DNA of an organism
and cloning it. Gene transfer is how scientists then insert it into the DNA of
another organism so that the genotype of that organism is altered to include
this gene.

{\bf Recombinant DNA} (rDNA) is the general name for taking a piece of one DNA,
and combining it with another strand of DNA. This new strand of DNA (now
called rDNA) can be introduced into a cell to alter the cell's genotype.

\section*{How it works}

The process of making a protein using rDNA technology involves five stages:

\begin{enumerate}

	\item {\bf Isolation} of the DNA fragments that have the gene for the
	desired protein (assuming you know where to find these fragments on the
	chromosome).

	\item {\bf Insertion} of the DNA fragments into a {\it vector}.

	\item {\bf Transformation} which is the transfer of the isolated DNA into a
	suitable host cell.

	\item {\bf Identification} of the host cells that have succesfully taken up
	the gene (by use of {\it gene markers}).

	\item {\bf Cloning} of the modified host cells to create a population of
	them.

\end{enumerate}

Each of these stages are far more complicated than this\marginpar{Except from
cloning, that's easy...}, and contain many substages and niggly bits, each will
be covered in greater detail here:

\subsection*{Isolation}

The aim of isolation is to extract the segment of DNA from a cell that codes for
the emergant (and desired) phenotypical characteristics.

\begin{enumerate}
	
	\item First, a cell that produces the protein we want to extract is isolated
	and some of it's messenger RNA (mRNA) is extracted.

	\item An enzyme called {\bf reverse transcriptase} is then used to make DNA
	from the isolated mRNA. This DNA is complementary to the mRNA, and so is
	referred to as complementary DNA (cDNA).

	\item The enzyme DNA polymerase is then used to build up the complementary
	nucleotides onto the cDNA strand to create a double strand of DNA that is
	the isolated gene.

\end{enumerate}

\subsection*{Insertion}

In order to make use of the gene we just extracted, we need to get it into the
cell that we want to genetically modify. This concerns both the {\it insertion}
and {\it transformation} stage. The insertion stage is principally concerned
with splicing the isolated gene into another piece of 'carrier' DNA. This
carrier DNA is called a {\it vector}.

\begin{enumerate}
	
	\item A restriction endonuclease enzyme is used to cut the obtained gene at
	points where there is a specific base sequence (such as AAGCTT). This
	creates a 'sticky end' where one strand of DNA carries on but the other is
	cut sooner.

	\item A plasmid of bacteria DNA is also cut using the same enzyme. This
	makes a pair of stick ends complementary to those on the gene.
	\marginpar{Note that the {\it plasmid} is the vector here.}

	\item DNA ligase is then used to splice the gene into the plasmid by joining
	the complementary ends of the DNA strands. The plasmids now have recombinant
	DNA.

\end{enumerate}

\textit{N.b. there are two types of restriction endonuclease, ones that cut a
'blunt' end straight through both DNA strands, and ones that cut 'sticky' or
'palendromic' ends where a cut is made in each strand, but they are a few bases
apart.}

\subsection*{Tranformation}

Transformation is when the DNA is transferred into the cells that are to be given the extra DNA. This process has a very low success rate (less than 1\%).

\begin{enumerate}
	
	\item The plasmids of DNA are inserted into the bacterium by immersing both
	in a solution of calcium ions and fluctuating the temperature.

	\item This makes the cell wall of the bacterium permiable which then allows
	plasmids to diffuse in.

\end{enumerate}

However, some plasmids (in the insertion stage) will have closed up without
incorperating the gene we want them to (since the ends of the plasmid we cut are
also complementary to eachother), and few bacteria will take up plasmids at all.

\subsection*{Identification}

Since very few cells {\it actually} take up the new gene, it is necessary to use
some sort of method to distinguish between cells where the gene transfer has
been successful (so they can be cloned) and those where the transfer wasn't
successful.

\begin{enumerate}
	
	\item Another gene is spliced into the plasmid (in the insertion stage) that
	will act as a marker to show the plasmid has been accepted.

	\item This can be a gene for anything from antibiotic resistance to a
	florescent protein.

	\item The cells that don't display the trait associated with the marker gene
	in their phenotype can then be disgarded.

\end{enumerate}

\section*{In vitro gene cloning}

The method I've just outlined is known as {\it in vivo} gene cloning (i.e. it
happens inside a living cell). However, there is another way of cloning genes
outside of a cell. This is called {\it in vitro} cloning.

The polymerase chain reaction (PCR) is a way of copying fragments of DNA. The process is automatic, so it's quick and efficient.

The PCR reaction has the following prerequisites:

\begin{itemize}
	
	\item The {\bf DNA fragment} to copy.

	\item An enzyme called {\bf DNA polymerase} which is capable of joining tens
	of thousands of nucleotides in a matter of minutes at high temperatures.

	\item {\bf Primers} which are short sequences of nucleotides that are
	complementary to the ends of the two DNA fragments.

	\item The four {\bf nucleotides} with which DNA can be created.

	\item A {\bf thermocycler} which is a machine that can control the
	temperature of a substance very precicely over time.

\end{itemize}

The PCR reaction contains three stages:

\begin{enumerate}

	\item The DNA fragments, nucleotides, primers and DNA polymerase are all
	placed in the thermocycler at around 95$^\circ$C which causes the two
	strands to unravel.

	\item The mixture is then cooled to 55$^\circ$C which allows the primers to
	join to their complementary bases on the DNA fragment. Their role is to
	provide a starting point for DNA polymerase to attach onto and begin adding
	nucleotides.

	\item The temperature is then increased to 72$^\circ$C which is the optimum
	for DNA polymerase. Two DNA polymerase enzymes work from each end of the
	DNA fragment until they meet in the middle and fall off, leaving a
	completed strand of DNA.

\end{enumerate}

The whole process takes about two minutes, and it can be repeated over and over
simply by having the thermocycler in a loop of these three temperatures so that
the DNA continues to be copied until the nucleotides run out.

\section*{The relative advantages of in vitro and in vivo DNA cloning}

\begin{tabular}{|p{5.6cm}|p{5.6cm}|}
	
	\hline

	{\bf Advantages of in vitro} & {\bf Advantages of in vivo} \\ \hline

	Extremely rapid & Allows us to introduce genes into organisms easily. \\
	\hline

	Doesn't require living cells & Low risk of contamination of DNA (since same
	endonucleases are used to match the 'sticky ends') \\ \hline

	& Very accurate with a low error rate. \\ \hline

	& Easy to copy a specific gene rather than the whole sample \\ \hline

	& Can produce transformed bacteria to produce large quantities of products.
	\\ \hline

\end{tabular}

\section*{Sample use cases of rDNA technology}
\subsection*{Insulin synthesis}

Insulin can be made by E.coli. Scientists isolated the gene for insulin
production and inserted it into a plasmid normally present in an E.coli
bacterium. The plasmid was then taken up by the E.coli cell. This meant that the
bacterium now started to secrete human insulin, which could be harvested. The
bacteria is then cultured in large fermenters that allow scientists to extract
large quantities of human insulin.

\subsection*{AAT synthesis}

AAT (alpha 1 antitrypsin protein) is a protein produced by humans that helps
protect body tissue from the harmful effects of other proteins. Some people have
a genetic defect that means that not enough AAT is produced, which leads to
emphysema due to enzymes destroying alveoli walls and reducing their elastcity.
Scientists added DNA that produces AAT to the embryo of a female sheep, which
they then cloned many times. The protein is then extracted from the milk of
these transformed sheep to use as a treatment for AAT deficiency induced
emphysema. {\it N.b. this won't treat smoking induced emphysema since the same
disease is brought about via a different cause.}

\subsection*{Cystic fibrosis treatment}
Cystic fibrosis is caused by a mutation in the human CTFR gene. The CTFR gene
codes for a protein that transports chloride across epithelial cell membranes.
When the CTFR gene is mutated, the tertiary structure of the CTFR protein
changes, so no chloride ions can be actively transported out of epithelial
cells. If chloride ions {\it were} transported (like they are in normal cells),
then the water potential of the lumen would decrease so that water would pass
out of the cell by osmosis. In people with cystic fibrosis this doesn't happen
since the CTFR protein doesn't work and the fluid in the lumen becomes very
viscous. 

It is possible to extract healthy CTFR genes from 'normal' people, but in order
to treat a patient with cystic fibrosis, these genes must be inserted into each
of the billions of cells that have CTFR proteins in the patients body. It is
possible to do that in a number of ways:
\begin{longtable}{p{\textwidth-20\tabcolsep-1in} p{\textwidth-20\tabcolsep-1in}}
Method & Issues\\ \midrule
\endfirsthead
Method & Issues\\ \midrule
\endhead
\midrule
\multicolumn{2}{r}{continued \ldots}
\endfoot
\endlastfoot
	Wrap healthy CTFR genes in lipids that are absorbed into the patient's
	cells through the cell wall. & Most genes that are absorbed aren't expressed
	by the cell, so few cells are made to function properly.\\ \midrule
	Insert healthy CTFR genes into harmless viruses that then infect the
	patient's epithelial cells and release their payload of CTFR gene into the
	cell. & The viruses released don't harm the cell or replicate (in theory,
	though there have been cases in trials where the viruses did act as
	pathogens), but there is only a slim chance that the deposited genes will
	get into the nucleus and start being used to create proteins.\\ \midrule
\end{longtable}

\subsection*{Gene therapy for SCID}

SCID (Severe Combined ImmunoDeficiency disease) is a rare inherited condition
that renders the immune system of children useless, which means that they are
susceptible to infections that while normally trivial for healthy children, can
develop into life threatening ilnesses.

Currently, the only technique used to combat SCID is to have the child wear an
'isolation bubble', to prevent exposure to pathogens.

The disease is brought about by the child having inherited a set of faulty
alleles, meaning that the enzyme {\it adenosine deaminase (ADA)} isn't produced.
In the absence of this enzyme, T lympocytes (which normally detect pathogens
that have entered the body) cannot function and are destroyed.

Gene therapy is being used to treat SCID. The aim of the treatment is to replace
the T lymphocytes in the child's body with ones that have the normal ADA gene.
This is acheived using the following method:

\begin{enumerate}

	\item A restriction endonuclease is used to isolate the normal ADA gene from
	the DNA of a healthy cell (by cutting the DNA when it meets a specific
	sequence of bases).

	\item The normal ADA gene is inserted into a virus, which is then allowed to
	replicate in a cell culture to produce large numbers of the virus.

	\item T lymphocytes are extracted from the patient's bone marrow and are
	cloned to increase their number.

	\item The viruses with the ADA gene are mixed with the (cloned) T
	lymphocytes of the patient. The viruses inject their payload of DNA into
	the nuclei of the T lymphocytes.

	\item The white blood cells with the 'normal' ADA gene are transfused back
	into the patient where they will undergo mitosis and provide a supply of
	white blood cells {\it with} the healthy ADA gene.

\end{enumerate}

%TODO:Add questions

\end{document}